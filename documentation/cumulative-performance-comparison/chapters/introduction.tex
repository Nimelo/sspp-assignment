\chapter{Introduction} \label{chp:introduction}
	\begin{comment}
		Introduce the following subordinate sections. This section identifies the issuing organization and the
		details of issuance. It includes required approvals and status (DRAFT/FINAL) of the document. It is
		here that the scope is described and references identified.
	\end{comment}

\section{Document identifier} \label{s:introduction:document-identifier}
	\begin{comment}
		Uniquely identify a version of the document by including information such as the date of issue, the
		issuing organization, the author(s), the approval signatures (possibly electronic), and the status/version
		(e.g., draft, reviewed, corrected, or final). Identifying information may also include the reviewers and
		pertinent managers. This information is commonly put on an early page in the document, such as the
		cover page or the pages immediately following it. Some organizations put this information at the end
		of the document. This information may also be kept in a place other than in the text of the document
		(e.g., in the configuration management system or in the header or footer of the document).
	\end{comment}
	\textsc{Document version: } 1.0 \\
	\textsc{Date of issue:} \today \\
	\textsc{Issuing organization:} \textit{Cranfield Universtity} \\
	\textsc{Authors: } \textit{Mateusz Gasior} \\
	\textsc{Status: } Final \\
\section{Scope} \label{s:introduction:scope}
	\begin{comment}
		Identify the test items (software or system) that are the object of testing, e.g., specific attributes of the
		software, the installation instructions, the user instructions, interfacing hardware, database conversion
		software that is not a part of the operational system) including their version/revision level. Also
		identify any procedures for their transfer from other environments to the test environment.
		Supply references to the test item documentation relevant to an individual level of test, if it exists, such
		as follows:
		⎯ Requirements
		⎯ Design
		⎯ User’s guide
		⎯ Operations guide
		⎯ Installation guide
		Reference any Anomaly Reports relating to the test items.
		Identify any items that are to be specifically excluded from testing.
	\end{comment}
	This document contains cumulative comparison between different implementations of matrix--vector dot product using following sparse matrix formats: \gls{CRS} and \gls{ELL} and technologies as: \gls{openmp} and \gls{cuda}.
	
	This document is derived from the \emph{Performance Test Case} document which chooses the best settings and tunning options computational results. Additionally part of the scope is also included in the performance \gls{TLD} document, mentioned in section \ref{s:introduction:references}. The values are presented in either \gls{FLOPS} or Double Precision \gls{FLOPS} depending on the precision of storage data type.
\section{References} \label{s:introduction:references}
	\begin{comment}
		List all of the applicable reference documents. The references are separated into “external” references
		that are imposed external to the project and “internal” references that are imposed from within to the
		project. This may also be at the end of the document.
	\end{comment}
	\begin{enumerate}
		\item \emph{Mateusz Gasior, \gls{SRS} - Computational Library}. Cranfield University, 2017.
		\item \emph{Mateusz Gasior, Performance Test Design - Computational Library}. Cranfield University, 2017.
		\item \emph{Mateusz Gasior, Performance Test Case - Computational Library}. Cranfield University, 2017.
		\item \emph{Mateusz Gasior, \gls{MTP} - Computational Library}. Cranfield University, 2017.
		\item \emph{University of Florida Sparse Matrix Collection}, \url{https://www.cise.ufl.edu/research/sparse/matrices/}, Last access April 2017.
	\end{enumerate}
\section{Context} \label{s:introduction:context}
	\begin{comment}
		Provide any required context that is not already covered by other sections of this document (e.g., thirdparty
		testing via the Internet).
	\end{comment}
	All the tested matrices come from the \emph{University of Florida Sparse Matrix Collection}.
\section{Notation for description} \label{s:introduction:notation-for-description}
	\begin{comment}
		Define any numbering schemes, e.g., for scenarios and test cases. The intent of this section is to
		explain any such schema.
	\end{comment}
	Notation of this document uses simple bar plots with natural language explanation, which is the most convenient way to describe obtained performance results.
	