\begin{testcase}{\emph{CRSTransformer} transform I}{High}{None}
		{
			\textbf{FN-4.1.2}
		}
		\term{Objective}{Transforming \gls{MM} format to \gls{CRS}}
		\term{Inputs}
		{
			Structure that uses a correct \gls{MM} header values followed by entries of the the sparse matrix. Example input is presented in Listing \ref{lst:crstransformer-1}. 	
		}
		\begin{lstlisting}[label={lst:crstransformer-1},
			    basicstyle=\small,caption={\gls{MM} format data example}, frame=single]
%%MatrixMarket matrix coordinate real general
4 4 7
1 1 11
1 2 12
2 2 22
2 3 23
3 3 33
4 3 43
4 4 44
		\end{lstlisting}
		\begin{lstlisting}[label={lst:crstransformer-1-sol},
		basicstyle=\small,caption={\gls{CRS} format internal data example}, frame=single]
VALUES = {11, 12, 22, 23, 33, 43, 44}
COLUMN_INDICES = {0, 1, 1, 2, 2, 2, 3}
ROW_START_INDEXES = {0, 2, 4, 5, 7}
		\end{lstlisting}
		\term{Steps}
		{
			\begin{enumerate}
				\item Prepare testing data-stream with example file.
				\item Load header and values using developed class.
				\item Invoke transforming routine 
				\item
				{
					Check internal data of \gls{CRS} structure comparing to values from Listing \ref{lst:crstransformer-1-sol} 
				}
			\end{enumerate}
		}
		\term{Output}{Class should correctly transform objects between formats.}
	\end{testcase}

\begin{testcase}{\emph{CRSTransformer} transform REAL GENERAL}{High}{None}
	{
		\textbf{FN-4.1.2}
	}
	\term{Objective}{Transforming \gls{MM} format to \gls{CRS}}
	\term{Inputs}
	{
		Structure that uses a correct \gls{MM} header values followed by entries of the the sparse matrix. Example input is presented in Listing \ref{lst:crstransformer-1-rg}. 	
	}
	\begin{lstlisting}[label={lst:crstransformer-1-rg},
	basicstyle=\small,caption={\gls{MM} format data example}, frame=single]
%%MatrixMarket matrix coordinate real general
4 4 7
1 1 11
1 2 12
2 2 22
2 3 23
3 3 33
4 3 43
4 4 44
	\end{lstlisting}
	\begin{lstlisting}[label={lst:crstransformer-rg-sol},
	basicstyle=\small,caption={\gls{CRS} format internal data example}, frame=single]
VALUES = {11, 12, 22, 23, 33, 43, 44}
COLUMN_INDICES = {0, 1, 1, 2, 2, 2, 3}
ROW_START_INDEXES = {0, 2, 4, 5, 7}
	\end{lstlisting}
	\term{Steps}
	{
		\begin{enumerate}
			\item Prepare testing data-stream with example file.
			\item Load header and values using developed class.
			\item Invoke transforming routine 
			\item
			{
				Check internal data of \gls{CRS} structure comparing to values from Listing \ref{lst:crstransformer-rg-sol} 
			}
		\end{enumerate}
	}
	\term{Output}{Class should correctly transform objects between formats.}
\end{testcase}

\begin{testcase}{\emph{CRSTransformer} transform PATTERN GENERAL}{High}{None}
	{
		\textbf{FN-4.1.2}
	}
	\term{Objective}{Transforming \gls{MM} format to \gls{CRS}}
	\term{Inputs}
	{
		Structure that uses a correct \gls{MM} header values followed by entries of the the sparse matrix. Example input is presented in Listing \ref{lst:crstransformer-pg}. 	
	}
	\begin{lstlisting}[label={lst:crstransformer-pg},
	basicstyle=\small,caption={\gls{MM} format data example}, frame=single]
%%MatrixMarket matrix coordinate pattern general
4 4 7
1 1 
1 2 
2 2 
2 3 
3 3 
4 3 
4 4 
	\end{lstlisting}
	\begin{lstlisting}[label={lst:crstransformer-pg-sol},
	basicstyle=\small,caption={\gls{CRS} format internal data example}, frame=single]
COLUMN_INDICES = {0, 1, 1, 2, 2, 2, 3}
ROW_START_INDEXES = {0, 2, 4, 5, 7}
	\end{lstlisting}
	\term{Steps}
	{
		\begin{enumerate}
			\item Prepare testing data-stream with example file.
			\item Load header and values using developed class.
			\item Invoke transforming routine 
			\item
			{
				Check internal data of \gls{CRS} structure comparing to values from Listing \ref{lst:crstransformer-pg-sol} 
			}
		\end{enumerate}
	}
	\term{Output}{Class should correctly transform objects between formats.}
\end{testcase}

\begin{testcase}{\emph{CRSTransformer} transform REAL SYMMETRIC}{High}{None}
	{
		\textbf{FN-4.1.2}
	}
	\term{Objective}{Transforming \gls{MM} format to \gls{CRS}}
	\term{Inputs}
	{
		Structure that uses a correct \gls{MM} header values followed by entries of the the sparse matrix. Example input is presented in Listing \ref{lst:crstransformer-rs}. 	
	}
	\begin{lstlisting}[label={lst:crstransformer-rs},
	basicstyle=\small,caption={\gls{MM} format data example}, frame=single]
%%MatrixMarket matrix coordinate real symmetric
4 4 7
1 1 11
1 2 12
2 2 22
2 3 23
3 3 33
4 3 43
4 4 44
2 1 12
3 2 23
3 4 43
	\end{lstlisting}
	\begin{lstlisting}[label={lst:crstransformer-rs-sol},
	basicstyle=\small,caption={\gls{CRS} format internal data example}, frame=single]
VALUES = {11, 12, 12, 22, 23, 23, 33, 43, 43, 44}
COLUMN_INDICES = {0, 1, 0, 1, 2, 1, 2, 3, 2, 3}
ROW_START_INDEXES = {0, 2, 5, 8, 10}
	\end{lstlisting}
	\term{Steps}
	{
		\begin{enumerate}
			\item Prepare testing data-stream with example file.
			\item Load header and values using developed class.
			\item Invoke transforming routine 
			\item
			{
				Check internal data of \gls{CRS} structure comparing to values from Listing \ref{lst:crstransformer-rs-sol} 
			}
		\end{enumerate}
	}
	\term{Output}{Class should correctly transform objects between formats.}
\end{testcase}

\begin{testcase}{\emph{CRSTransformer} transform PATTERN SYMMETRIC}{High}{None}
	{
		\textbf{FN-4.1.2}
	}
	\term{Objective}{Transforming \gls{MM} format to \gls{CRS}}
	\term{Inputs}
	{
		Structure that uses a correct \gls{MM} header values followed by entries of the the sparse matrix. Example input is presented in Listing \ref{lst:crstransformer-ps}. 	
	}
	\begin{lstlisting}[label={lst:crstransformer-ps},
	basicstyle=\small,caption={\gls{MM} format data example}, frame=single]
%%MatrixMarket matrix coordinate pattern symmetric
4 4 7
1 1 
1 2 
2 2 
2 3 
3 3 
4 3 
4 4 
2 1 
3 2 
3 4 
	\end{lstlisting}
	\begin{lstlisting}[label={lst:crstransformer-ps-sol},
	basicstyle=\small,caption={\gls{CRS} format internal data example}, frame=single]
COLUMN_INDICES = {0, 1, 0, 1, 2, 1, 2, 3, 2, 3}
ROW_START_INDEXES = {0, 2, 5, 8, 10}
	\end{lstlisting}
	\term{Steps}
	{
		\begin{enumerate}
			\item Prepare testing data-stream with example file.
			\item Load header and values using developed class.
			\item Invoke transforming routine 
			\item
			{
				Check internal data of \gls{CRS} structure comparing to values from Listing \ref{lst:crstransformer-ps-sol} 
			}
		\end{enumerate}
	}
	\term{Output}{Class should correctly transform objects between formats.}
\end{testcase}