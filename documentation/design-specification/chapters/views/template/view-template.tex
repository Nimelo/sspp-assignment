\chapter{View Template} \label{chp:view-template}
	\begin{comment}
		An SDD shall be organized into one or more design views.
		
		Each design view in an SDD shall conform to its governing design viewpoint.
		The purpose of a design view is to address design concerns pertaining to the design subject, to allow a
		design stakeholder to focus on design details from a specific perspective and effectively address relevant
		requirements.
		
		Each design view shall address the design concerns specified by its governing design viewpoint.
		
		An SDD is complete when each identified design concern is the topic of at least one design view; all design
		attributes refined from each design concern by some viewpoint have been specified for all of the design
		entities and relationships in its associated view; and all design constraints have been applied.
		
		An SDD is consistent if there are no known conflicts between the design elements of its design views.
		
		NOTE—Users of this standard may wish to state delivery requirements on an SDD in terms of the above notions of
		completeness and consistency.
	\end{comment}
	
	%TABLE WITH NAME OF VIEWPOINT, PURPOSE, REQUIREMENTS
		
	\section{Design elements} \label{s:view-template:design-elements}
		\begin{comment}
			A design element is any item occurring in a design view. A design element may be any of the following
			subcases: design entity, design relationship, design attribute, or design constraint.
			
			Each design element in the SDD shall have a name (4.6.2.1), a type (4.6.2.2), and any contents.
			
			NOTE 1—This requirement is “inherited” by the four subcases: design entities, design relationships, design attributes,
			and design constraints.
			
			The type of each design element shall be introduced within exactly one design viewpoint definition.
			
			A design element may be used in one or more design views.
			
			NOTE 2—A design element is introduced and “owned” by exactly one design view, in accordance with its type
			definition within the associated viewpoint. It may be shared or referenced within other design views. Sharing of design
			elements permits the expression of design aspects as in aspect-oriented design.
		\end{comment}
		
		\begin{design-element}{Name}{Type}
			content...
		\end{design-element}
	\section{Design entities} \label{s:view-template:design-entities}
		\begin{comment}
			Design entities capture key elements of a software design.
			
			Each design entity shall have a name (4.6.2.1), a type (4.6.2.2), and purpose (4.6.2.3).
			
			Examples of design entities include, but are not limited to, the following: systems, subsystems, libraries,
			frameworks, abstract collaboration patterns, generic templates, components, classes, data stores, modules,
			program units, programs, and processes.
			
			NOTE—The number and types of entities needed to express a design view are dependent on a number of factors, such
			as the complexity of the system, the design technique used, and the tool support environment.
		\end{comment}
		
		\begin{design-entity}{Name}{Type}{Purpose}
			content...
		\end{design-entity}
		
	\section{Design overlays} \label{s:view-template:design-overlays}
		\begin{comment}
			A design overlay is used for presenting additional information with respect to an already-defined design
			view.
			
			Each design overlay shall be uniquely named and marked as an overlay.
			
			Each design overlay shall be clearly associated with a single viewpoint.
			
			NOTE—Reasons to utilize a design overlay as a part of an SDD include: to provide an extension mechanism for design
			information to be presented conveniently on top of some view without a requirement for existing external
			standardization of languages and notations for such representation; to extend expressive power of representation with
			additional details while reusing information from existing views (i.e., without a need to define additional views or
			persistently store derivable design information); and to relate design information with facts from the system
			environment for the convenience of the designer (or other stakeholders).
		\end{comment}	
		
	
	\section{Design Rationale} \label{s:view-template-design-rationale}
		\begin{comment}
			Design rationale captures the reasoning of the designer that led to the system as designed and the
			justifications of those decisions.
		
			Design rationale may take the form of commentary, made throughout the decision process and associated
			with collections of design elements. Design rationale may include, but is not limited to: design issues raised
			and addressed in response to design concerns; design options considered; trade-offs evaluated; decisions
			made; criteria used to guide design decisions; and arguments and justifications made to reach decisions.
		
			NOTE—The only required design rationale is use of the purpose attribute (4.6.2.3).
		\end{comment}
		
	\section{Design languages} \label{s:view-template:design-languages}
		\begin{comment}
			Design languages are selected as a part of design viewpoint specification (4.5).
		
			A design language may be selected for a design viewpoint only if it supports all design elements defined by
			that viewpoint.
		
			Design languages shall be selected that have:
			⎯ A well-defined syntax and semantics; and
			⎯ The status of an available standard or equivalent defining document.
			
			Only standardized and well-established (i.e., previously defined and conveniently available) design
			languages shall be used in an SDD. In the case of a newly invented design language, the language
			definition must be provided as a part of the viewpoint declaration.
		
			NOTE 1—Standardized design languages that are in common use are preferable to established languages without a
			formal definition. Examples of standardized languages include: IDEF0 (IEEE Std 1320.1™-1998 [B18]); IDEF1X
			(IEEE Std 1320.2™-1998 [B19]); Unified Modeling Language (UML) (OMG [B28] and [B29]); Vienna Definition
			Method (VDM) (ISO/IEC 13817-1:1996 [B24]); and Z (ISO/IEC 13568:2002 [B23]). Examples of established
			languages include: state machines, automata, decision tables, Warnier diagrams, Jackson Structured Design (JSD),
			program design languages (PDL), structure charts, Hierarchy plus Input-Process-Output (HIPO), reliability models, and
			queuing models.
		
			NOTE 2—It is acceptable to use a design language in more than one view. It is also acceptable to use more than one
			design language within any number of views when each design language to be used is declared by the viewpoint. This
			is acceptable even for a portion of the design; for example, when used as a basis for interchange; due to organizational
			considerations such as development by non-collocated team members; subcontracting of partial design responsibility;
			or taking advantage of particular design tools or designer expertise.
		
			NOTE 3—Annex B establishes a uniform format for describing design languages to be used in SDDs.
			
			NOTE 4—In case that no adequate design language is readily available for a specified viewpoint, it is the designer’s
			responsibility to provide an adaptation of an existing language or the definition of an appropriate new design language.
			This design language definition would be provided by the designer to be included in the SDD in accordance with the
			requirements for viewpoints in 4.5.
		\end{comment}