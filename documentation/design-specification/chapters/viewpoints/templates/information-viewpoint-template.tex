\chapter{Information Viewpoint Template} \label{chp:information-viewpoint-template}
	\begin{comment}
		The Information viewpoint is applicable when there is a substantial persistent data content expected with
		the design subject.
	\end{comment}
	
	\section{Design concerns} \label{s:information-viewpoint-template:design-concerns}
		\begin{comment}
			Key concerns include persistent data structure, data content, data management strategies, data access
			schemes, and definition of metadata.
		\end{comment}
	
	\section{Design elements} \label{s:information-viewpoint-template:design-elements}
		\begin{comment}
			Design entities: data items, data types and classes, data stores, and access mechanisms.
			
			Design relationships: association, uses, implements. Data attributes, their constraints and static
			relationships among data entities, aggregates of attributes, and relationships.
			
			Design attributes: persistence and quality properties.
		\end{comment}
	
		\subsection{Data attribute} \label{s:information-viewpoint-template:data-attribute}
			\begin{comment}
				A description of data elements internal to the entity. The data attribute describes the method of
				representation, initial values, use, semantics, format, and acceptable values of internal data. The description
				of data may be in the form of a data dictionary that describes the content, structure, and use of all data
				elements. Data information should describe everything pertaining to the use of data or internal data
				structures by this entity. It should include data specifications such as formats, number of elements, and
				initial values. It should also include the structures to be used for representing data such as file structures,
				arrays, stacks, queues, and memory partitions.
				
				The meaning and use of data elements should be specified. This description includes such things as static
				versus dynamic, whether it is to be shared by transactions, used as a control parameter, or used as a value,
				loop iteration count, pointer, or link field. In addition, data information should include a description of data
				validation needed for the process.
				
				NOTE—This design attribute is retained for compatibility with IEEE Std 1016-1998.
			\end{comment}
			
	\section{Example languages} \label{s:information-viewpoint-template:example-languages}
		\begin{comment}
			IDEF1X (IEEE Std 1320.2™-1998 [B19]), UML class diagrams (OMG [B28]).
		\end{comment}