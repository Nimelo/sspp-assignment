\chapter{Structure Viewpoint Template} \label{chp:structure-viewpoint-template}
	\begin{comment}
		The Structure viewpoint is used to document the internal constituents and organization of the design subject
		in terms of like elements (recursively).
	\end{comment}
	
	\section{Design concerns} \label{s:structure-viewpoint-template:design-concerns}
		\begin{comment}
			Compositional structure of coarse-grained components and reuse of fine-grained components.
		\end{comment}
	
	\section{Design elements} \label{s:structure-viewpoint-template:design-elements}
		\begin{comment}
			Design entities: port, connector, interface, part, and class.
			
			Design relationships: connected, part of, enclosed, provided, and required.
			
			Design attributes: name, type, purpose, and definition.
			
			Design constraints: interface constraints, reusability constraints, and dependency constraints.
		\end{comment}
	
	\section{Example languages} \label{s:structure-viewpoint-template:example-languages}
		\begin{comment}
			UML composite structure diagram, UML class diagram, and UML package diagram (OMG [B28]).
		\end{comment}