\chapter{Interface Viewpoint Template} \label{chp:interface-viewpoint-template}
	\begin{comment}
		The Interface viewpoint provides information designers, programmers, and testers the means to know how
		to correctly use the services provided by a design subject. This description includes the details of external
		and internal interfaces not provided in the SRS. This viewpoint consists of a set of interface specifications
		for each entity.
		NOTE—User interfaces are addressed separately.
	\end{comment}
	
	\section{Design concerns} \label{s:interface-viewpoint-template:design-concerns}
		\begin{comment}
			An Interface view description serves as a binding contract among designers, programmers, customers, and
			testers. It provides them with an agreement needed before proceeding with the detailed design of entities.
			The interface description is used by technical writers to produce customer documentation or may be used
			directly by customers. In the latter case, the interface description could result in the production of a human
			interface view.
			
			Designers, programmers, and testers often use design entities that they did not develop. These entities can
			be reused from earlier projects, contracted from an external source, or produced by other developers. The
			interface description establishes an agreement among designers, programmers, and testers about how
			cooperating entities will interact. Each entity interface description should contain everything another
			designer or programmer needs to know to develop software that interacts with that entity. A clear
			description of entity interfaces is essential on a multi-person development for smooth integration and ease
			of maintenance.
		\end{comment}
	
	\section{Design elements} \label{s:interface-viewpoint-template:design-elements}
		\begin{comment}
			The attributes for identification (4.6.2.1), function (5.3.2.1), and interface (5.8.2.1) should be provided for
			all design entities.
		\end{comment}
	
		\subsection{Interface attribute} \label{s:interface-viewpoint-template:interface-attribute}
			\begin{comment}
				A description of how other entities interact with this entity. The interface attribute describes the methods of
				interaction and the rules governing those interactions. Methods of interaction include the mechanisms for
				invoking or interrupting the entity, for communicating through parameters, common data areas or
				messages, and for direct access to internal data. The rules governing the interaction include the
				communications protocol, data format, acceptable values, and the meaning of each value.
				
				This attribute provides a description of the input ranges, the meaning of inputs and outputs, the type and
				format of each input or output, and output error codes. For information systems, it should include inputs,
				screen formats, and a complete description of the interactive language.
				
				NOTE—This design attribute is retained for compatibility with IEEE Std 1016-1998
			\end{comment}
			
	\section{Example languages} \label{s:interface-viewpoint-template:example-languages}
		\begin{comment}
			Interface definition languages (IDL), UML component diagram (OMG [B28]). In case of user interfaces
			the Interface view should include screen formats, valid inputs, and resulting outputs. For data-driven
			entities, a data dictionary should be used to describe the data characteristics. Those entities that are highly
			visible to a user and involve the details of how the customer should perceive the system should include a
			functional model, scenarios for use, detailed feature sets, and the interaction language.
		\end{comment}