\chapter{Algorithm Viewpoint}
\label{ch:algorithm-viewpoint}
\begin{comment}
	The detailed design description of operations (such as methods and functions), the internal details and logic of each design entity.
\end{comment}
	
	\section{Design concerns}
	\label{sec:algorithm-viewpoint:design-concerns}
	The Algorithm viewpoint provides details of algorithms in regard to time-space performance and processing logic prior to implementation, and to aid in producing unit test plans.
	\begin{comment}
		The Algorithm viewpoint provides details needed by programmers, analysts of algorithms in regard to	time-space performance and processing logic prior to implementation, and to aid in producing unit test plans.
	\end{comment}
	
	\begin{concerns}{Transforming \gls{MM} to \gls{CRS}}{Structure}
		Below pseudocode assumes, that \texttt{tuples} is an array of \gls{MM} entry, containing row and column coordinates, as well as value of the entry. 
		\begin{lstlisting}[language=C++,caption={Pseudocode of transforming algorithm from \gls{MM} to \gls{CRS} format.}]
crs function transform_to_CRS()
	sort tuples by row and within each row by column;
	values[non_zeros];
	row_start_indexes[rows + 1];
	columnd_indices[non_zeros];
	set row_start_indexes_index to 0;
	set non_zeros_index to 1;
	set row_start_indexes[row_start_indexes_index++] to 0;
	
	while (row_start_indexes_index < rows + 1 and non_zeros_index < non_zeros) do
		begin
			set column_indices[non_zeros_index - 1] to tuples[non_zeros_index - 1]->column_indice;
			set values[non_zeros_index - 1] to tuples[non_zeros_index - 1]->value;
			set row_indices_diff to tuples[non_zeros_index].row_indice - tuples[non_zeros_index - 1].row_indice;
			if (row_indices_diff != 0) then
				set row_start_indexes[row_start_indexes_index++] to non_zeros_index;
				if (row_indices_diff > 1) then
					for (i = 0; i < row_indices_diff - 1; ++i) do
						set row_start_indexes[row_start_indexes_index++] to non_zeros_index;
					end for
				end if
			end if
			set non_zeros_index to non_zeros_index + 1;
		end
	for (i = row_start_indexes_index; i < rows; ++i) do 
		set row_start_indexes[row_start_indexes_index++] to non_zeros_index;
	end for
	set row_start_indexes[row_start_indexes_index++] to non_zeros_index;
	set column_indices[non_zeros_index - 1] to tuples[non_zeros_index - 1]->column_indice;
	set values[non_zeros_index - 1] to tuples[non_zeros_index - 1]->value;
	return crs(rows, columns, non_zeros, row_start_indexes, column_indices, values);
end function
		\end{lstlisting}
	\end{concerns}

	\begin{concerns}{Transforming \gls{MM} to \gls{ELL}}{}
		Below pseudocode assumes, that \texttt{tuples} is an array of \gls{MM} entry, containing row and column coordinates, as well as value of the entry.
		\begin{lstlisting}[language=C++, caption={Pseudocode of transforming algorithm from \gls{MM} to \gls{ELL} format.}]
ellpack function transform_to_ELLPACK()
	create aux with max_nonzeros per row in entries;
	find row in aux_vector with maximum non_zeros and save to max_non_zeros;
	values[max_non_zeros * rows];
	column_indices[max_non_zeros * rows];
	
	for (i = 0; i < non_zeros; i++) do
		set index to max_row_non_zeros * tuples[i]->row_indice + aux_vector[tuples[i]->row_indice] - 1;
		set column_indices[index] to tuples->column_indice;
		set values[index] to tuples->value;
		--aux_vector[tuples[i]->row_indice];
	end for
	
	return ellpack(rows, columns, non_zeros, max_non_zeros, column_indices, values);
end function
		\end{lstlisting}
	\end{concerns}

	\begin{concerns}{Calculations of \gls{CRS}--vector dot product in serial}{}
		\begin{lstlisting}[language=C++, caption={Pseudocode of \gls{CRS}--vector dot product in serial.}]
void function sequential_CRS()
	for (i = 0; i < rows; i++) {
		set tmp to 0;
		for (j = row_start_indexes[i]; j < row_start_indexes[i + 1]; j++) do
			set tmp to tmp + values[j] * vector[column_indices[j]];
		end for
		set result[i] to tmp;
	}
end function
		\end{lstlisting}
	\end{concerns}
	\clearpage
	\begin{concerns}{Calculations of \gls{ELL}--vector dot product in serial}{}
		\begin{lstlisting}[language=C++, caption={Pseudocode of \gls{ELL}--vector dot product in serial.}]
void function sequential_ELLPACK() 
	for (i = 0; i < rows; i++) do
		set tmp to 0;
		for (j = 0; j < max_non_zeros; j++) do
			set tmp to tmp + values[i][j] * vector[column_indices[i][j]];
		end for
		set result[i] to tmp;
	end for
end function
		\end{lstlisting}
	\end{concerns}	

	\begin{concerns}{Calculations of \gls{CRS}--vector dot product with \gls{openmp}}{}
		
		For effective usage of threads, each thread should be assigned with equal number of operations to perform. Following solution distributes equally chunks of rows between active threads. This solution is the most efficient, if each row in matrix contains same number of nonzeros. In worst case scenario, if nonzeros are not equally distributed in matrix, this algorithm can assign whole load to a single thread, causing other threads to wait. 
		\begin{lstlisting}[language=C++, caption={Pseudocode of \gls{CRS}--vector dot product with \gls{openmp}.}]
void function parallel()
	set threadId to current_thread_id;
	set threads to number_of_threads;
	set lowerBoundary to rows * threadId / threads;
	set upperBoundary to rows * (threadId + 1) / threads;
	#execute in parallel
	for (i = lowerBoundary; i < upperBoundary; i++) do
		set sum to 0;
		for (j = row_start_indexes[i]; j < row_start_indexes[i + 1]; j++) do
			set sum to sum + values[j] * vector[column_indices[j]];
		end for
		set result[i] to sum;
	end for
end function
		\end{lstlisting}
	\end{concerns}
	\clearpage
	\begin{concerns}{Calculations of \gls{ELL}--vector dot product with \gls{openmp}}{}
		
		This algorithm for ELLPACK format distributes work between threads the same way as preceding pseudocode for CRS.
		\begin{lstlisting}[language=C++, caption={Pseudocode of \gls{ELL}--vector dot product with \gls{openmp}.}]
void function parallel() 
	set threadId to current_thread_id;
	set threads to number_of_threads;
	set lowerBoundary to rows * threadId / threads;
	set upperBoundary to rows * (threadId + 1) / threads;
	#execute in parallel
	for (i = low; i < up; i++) do
		set sum to 0;
		for (j = 0; j < max_non_zeros; j++) do
			set sum to sum + values[i][j] * vector[column_indices[i][j]];
		end for
		set result[i] to sum;
	end for
end function
		\end{lstlisting}
	\end{concerns}	

	\pagebreak

	\begin{concerns}{Calculations of \gls{CRS}--vector dot product with \gls{cuda}}
		
		Host side driver routine dynamically determines number of blocks, based on number of rows in matrix. 
		
		\begin{lstlisting}[language=C++, caption={Pseudocode of host-side driver routine for \gls{ELL}--vector product.}]
void function host_driver(cudaDeviceProp& gpu_prop, ...)
set T to gpu_prop.maxThreadsPerBlock;
auto blocks = ceil(rows / T);
ellpack_kernel <<<blocks, T>>>(...);
end function
		\end{lstlisting}
		
		
\begin{lstlisting}[language=C++, caption={Pseudocode of \gls{cuda} kernel for CRS--vector dot product.}]
void function crs_kernel()
	set row to blockDim.x * blockIdx.x + threadIdx.x;
	if (row < rows)
		set dot to 0;
		set row_start to row_start_indexes[row];
		set row_end to row_start_indexes[row + 1];
		for (i = row_start; i < row_end; i++) do
			set dot to dot + values[i] * vector[column_indices[i]];
		end for
		set result[row] to dot;
	end if
end function
\end{lstlisting}
	\end{concerns}
	
	\pagebreak
	
	\begin{concerns}{Calculations of \gls{ELL}--vector dot product with \gls{cuda}}{}

		Host side driver routine dynamically determines number of blocks, based on number of rows in matrix. 
		
		\begin{lstlisting}[language=C++, caption={Pseudocode of host-side driver routine for \gls{ELL}--vector product.}]
void function host_driver(cudaDeviceProp& gpu_prop, ...)
	set T to gpu_prop.maxThreadsPerBlock;
	auto blocks = ceil(rows / T);
	ellpack_kernel <<<blocks, T>>>(...);
end function
		\end{lstlisting}
		
		This kernel assigns calculations of a single matrix row to a thread inside a block.
		
		\begin{lstlisting}[language=C++, caption={Pseudocode of \gls{cuda} kernel for \gls{ELL}--vector dot product.}]
void function ellpack_kernel()
	set row to blockDim.x * blockIdx.x + threadIdx.x;
	if (row < rows) then
		set dot to 0;
		for (n = 0; n < max_row_non_zeros; n++) do
			set index to row * max_row_non_zeros + n;
			set column to column_indices[index];
			set value to values[index];
			if (value != 0) then
				set dot to dot + value * vector[column];
			end if
		end for
		set result[row] to dot;
	end if
end function
		\end{lstlisting}
	\end{concerns}	