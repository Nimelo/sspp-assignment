\chapter{Interaction Viewpoint Template} \label{chp:interaction-viewpoint-template}
	\begin{comment}
		The Interaction viewpoint defines strategies for interaction among entities, regarding why, where, how, and
		at what level actions occur.
	\end{comment}
	
	\section{Design concerns} \label{s:interaction-viewpoint-template:design-concerns}
		\begin{comment}
			For designers. this includes evaluating allocation of responsibilities in collaborations, especially when
			adapting and applying design patterns; discovery or description of interactions in terms of messages among
			affected objects in fulfilling required actions; and state transition logic and concurrency for reactive,
			interactive, distributed, real-time, and similar systems.
		\end{comment}
	
	\section{Design elements} \label{s:interaction-viewpoint-template:design-elements}
		\begin{comment}
			Classes, methods. states, events, signals, hierarchy, concurrency, timing, and synchronization.
		\end{comment}
	
	\section{Example languages} \label{s:interaction-viewpoint-template:example-languages}
		\begin{comment}
			UML composite structure diagram, UML interaction diagram (OMG [B28]).
		\end{comment}