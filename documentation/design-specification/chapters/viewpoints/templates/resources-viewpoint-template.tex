\chapter{Resources Viewpoint Template} \label{chp:resources-viewpoint-template}
	\begin{comment}
		The purpose of the Resource viewpoint is to model the characteristics and utilization of resources in a
		design subject.
	\end{comment}
	
	\section{Design concerns} \label{s:resources-viewpoint-template:design-concerns}
		\begin{comment}
			Key concerns include resource utilization, resource contention, availability, and performance.
		\end{comment}
	
	\section{Design elements} \label{s:resources-viewpoint-template:design-elements}
		\begin{comment}
			Design entities: resources, usage policies.
			
			Design relationships: allocation and uses.
			
			Design attributes: identification (4.6.2.1), resource (5.13.2.1), performance measures (such as throughput,
			rate of consumption).
			
			Design constraints: priorities, locks, resource constraints.
		\end{comment}
		
		\subsection{Resource attribute} \label{s:resources-viewpoint-template:resource-attribute}
			\begin{comment}
				A description of the elements used by the entity that are external to the design. The resources attribute
				identifies and describes all of the resources external to the design that are needed by this entity to perform
				its function. The interaction rules and methods for using the resource are to be specified by this attribute.
			
				This attribute provides information about items such as physical devices (printers, disc-partitions, memory
				banks), software services (math libraries, operating system services), and processing resources (CPU
				cycles, memory allocation, buffers).
				
				The resources attribute should describe usage characteristics such as the processing time at which resources
				are to be acquired and sizing to include quantity, and physical sizes of buffer usage. It should also include
				the identification of potential race and deadlock conditions as well as resource management facilities.

				NOTE—This design attribute is retained for compatibility with IEEE Std 1016-1998.
			\end{comment}
			
	\section{Example languages} \label{s:resources-viewpoint-template:example-languages}
		\begin{comment}
			Woodside [B37], UML class diagram, UML Object Constraint Language (OMG [B28]).
		\end{comment}