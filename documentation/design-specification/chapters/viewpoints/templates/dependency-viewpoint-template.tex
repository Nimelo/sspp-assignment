\chapter{Dependency Viewpoint Template} \label{chp:dependency-viewpoint-template}
	\begin{comment}
		The Dependency viewpoint specifies the relationships of interconnection and access among entities. These
		relationships include shared information, order of execution, or parameterization of interfaces.
	\end{comment}
	
	\section{Design concerns} \label{s:dependency-viewpoint-template:design-concerns}
		\begin{comment}
			A Dependency view provides an overall picture of the design subject in order to assess the impact of
			requirements or design changes. It can help maintainers to isolate entities causing system failures or
			resource bottlenecks. It can aid in producing the system integration plan by identifying the entities that are
			needed by other entities and that must be developed first. This description can also be used by integration
			testing to aid in the production of integration test cases.
		\end{comment}
	
	\section{Design elements} \label{s:dependency-viewpoint-template:design-elements}
		\begin{comment}
			Design entities: subsystem, component, and module.
			
			Design relationships: uses, provides, and requires.
			
			Design attribute: name (4.6.2.1), type (4.6.2.2), purpose (4.6.2.3), dependencies (5.5.2.1), and resources.
			These attributes should be provided for all design entities.
		\end{comment}
	
		\subsection{Dependencies attribute} \label{s:dependency-viewpoint-template:dependencies-attribute}
			\begin{comment}
				A description of the relationships of this entity with other entities. The dependencies attribute identifies the
				uses or requires the presence of relationship for an entity. This attribute is used to describe the nature of
				each interaction including such characteristics as timing and conditions for interaction. The interactions
				involve the initiation, order of execution, data sharing, creation, duplicating, usage, storage, or destruction
				of entities.
				NOTE—This design entity attribute is retained for compatibility with IEEE Std 1016-1998.
			\end{comment}
			
	\section{Example languages} \label{s:dependency-viewpoint-template:example-languages}
		\begin{comment}
			UML component diagrams and UML package diagrams showing dependencies among subsystems (OMG
			[B28]).
		\end{comment}