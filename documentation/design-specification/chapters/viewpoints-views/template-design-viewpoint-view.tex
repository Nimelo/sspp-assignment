\chapter{Template Design Viewpoint--View} \label{chp:template-vpv}
	\section{Viewpoint} \label{s:template-vpv-viewpoint}		
		\subsection{Context viewpoint} \label{s:template-vpv:context-viewpoint}
			\begin{comment}
				The Context viewpoint depicts services provided by a design subject with reference to an explicit context.
				That context is defined by reference to actors that include users and other stakeholders, which interact with
				the design subject in its environment. The Context viewpoint provides a “black box” perspective on the
				design subject.
				Services depict an inherently functional aspect or anticipated cases of use of the design subject (hence “use
				cases” in UML). Stratification of services and their descriptions in the form of scenarios of actors’
				interactions with the system provide a mechanism for adding detail. Services may also be associated with
				actors through information flows. The content and manner of information exchange with the environment
				implies additional design information and the need for additional viewpoints (see 5.10).
				A Deployment overlay to a Context view can be transformed into a Deployment view whenever the
				execution hardware platform is part of the design subject; for stand-alone software design, a Deployment
				overlay maps software entities onto externally available entities not subject of the current design effort.
				Similarly, work allocation to teams and other management perspectives are overlays in the design.
			\end{comment}
			
			\subsubsection{Design concerns} \label{s:template-vpv:design-concerns}
				\begin{comment}
					The purpose of the Context viewpoint is to identify a design subject’s offered services, its actors (users and
					other interacting stakeholders), to establish the system boundary and to effectively delineate the design
					subject’s scope of use and operation.
					Drawing a boundary separating a design subject from its environment, determining a set of services to be
					provided, and the information flows between design subject and its environment, is typically a key design
					decision. That makes this viewpoint applicable to most design efforts.
					When the system is portrayed as a black box, with internal decisions hidden, the Context view is often a
					starting point of design, showing what is to be designed functionally as the only available information
					about the design subject: a name and an associated set of externally identifiable services. Requirements
					analysis identifies these services with the specification of quality of service attributes, henceforth invoking
					many non-functional requirements. Frequently incomplete, a Context view is begun in requirements
					analysis. Work to complete this view continues during design.
				\end{comment}
				
			\subsubsection{Design elements} \label{s:template-vpv:design-elements}
				\begin{comment}
					Design entities: actors—external active elements interacting with the design subject, including users, other
					stakeholders and external systems, or other items; services—also called use cases; and directed information
					flows between the design subject, treated as a black box, and its actors associating actors with services.
					Flows capture the expected information content exchanged.
					Design relationships: receive output and provide input (between actors and the design subject).
					Design constraints: qualities of service; form and medium of interaction (provided to and received from)
					with environment.
				\end{comment}
				
			\subsubsection{Example languages} \label{s:template-vpv:example-languages}
				\begin{comment}
					Any black-box type diagrams can be used to realize the Context viewpoint. Appropriate languages include
					Structured Analysis [e.g., IDEF0 (IEEE Std 1320.1-1998 [B18]), Structured Analysis and Design 
					Technique (SADT) (Ross [B32]) or those of the DeMarco or Gane-Sarson variety], the Cleanroom’s black
					box diagrams, and UML use cases (OMG [B28]).
				\end{comment}
			
			
	\section{Composition viewpoint} \label{s:tempalte-vpv:composition-viewpoint}
		\begin{comment}
			The Composition viewpoint describes the way the design subject is (recursively) structured into constituent
			parts and establishes the roles of those parts.
		\end{comment}
		
		\subsection{Design concerns} \label{s:tempalte-vpv:design-concerns}
			\begin{comment}
				Software developers and maintainers use this viewpoint to identify the major design constituents of the
				design subject, to localize and allocate functionality, responsibilities, or other design roles to these
				constituents. In maintenance, it can be used to conduct impact analysis and localize the efforts of making
				changes. Reuse, on the level of existing subsystems and large-grained components, can be addressed as
				well. The information in a Composition view can be used by acquisition management and in project
				management for specification and assignment of work packages, and for planning, monitoring, and control
				of a software project. This information, together with other project information, can be used in estimating
				cost, staffing, and schedule for the development effort. Configuration management may use the information
				to establish the organization, tracking, and change management of emerging work products (see
				IEEE Std 12207-2008 [B21]).
			\end{comment}
			
		\subsection{Design elements} \label{s:template-vpv:design-elements}
			\begin{comment}
				Design entities: types of constituents of a system: subsystems, components, modules; ports and (provided
				and required) interfaces; also libraries, frameworks, software repositories, catalogs, and templates.
				Design relationships: composition, use, and generalization. The Composition viewpoint supports the
				recording of the part-whole relationships between design entities using realization, dependency,
				aggregation, composition, and generalization relationships. Additional design relationships are required and
				provided (interfaces), and the attachment of ports to components.
				Design attributes: For each design entity, the viewpoint provides a reference to a detailed description via
				the identification attribute. The attribute descriptions for identification, type, purpose, function, and
				definition attribute should be utilized.
			\end{comment}
		
			\subsubsection{Function attribute} \label{s:template-vpv:function-attribute}
				\begin{comment}
					A statement of what the entity does. The function attribute states the transformation applied by the entity to
					its inputs to produce the output. In the case of a data entity, this attribute states the type of information
					stored or transmitted by the entity.
					NOTE—This design attribute is retained for compatibility with IEEE Std 1016-1998.
				\end{comment}
			
			\subsubsection{Subordinates attribute} \label{s:template-vpv:subordinates-attribute}
				\begin{comment}
					The identification of all entities composing this entity. The subordinates attribute identifies the “composed
					of” relationship for an entity. This information is used to trace requirements to design entities and to
					identify parent/child structural relationships through a design subject.
					NOTE—This design attribute is retained for compatibility with IEEE Std 1016-1998. An equivalent capability is
					available through the composition relationship.
				\end{comment}
			
		\subsection{Example languages} \label{s:template-vpv:example-languages}
			\begin{comment}
				UML component diagrams (see OMG [B28]) cover this viewpoint. The simplest graphical technique used
				to describe functional system decomposition is a hierarchical decomposition diagram; such diagram can be
				used together with natural language descriptions of purpose and function for each entity, such as is
				provided by IDEF0 (IEEE Std 1320.1-1998 [B18]), the Structure Chart (Yourdon and Constantine [B38],
				and the HIPO Diagram. Run-time composition can also use structured diagrams (Page-Jones [B29]).
			\end{comment}
		
	\section{Logical viewpoint} \label{s:template-vpv:logiacl-viewpoint}
		\begin{comment}
			The purpose of the Logical viewpoint is to elaborate existing and designed types and their implementations
			as classes and interfaces with their structural static relationships. This viewpoint also uses examples of
			instances of types in outlining design ideas.
		\end{comment}
		
		\subsection{Design concerns} \label{s:template-vpv:design-concerns}
			\begin{comment}
				The Logical viewpoint is used to address the development and reuse of adequate abstractions and their
				implementations. For any implementation platform, a set of types is readily available for the domain
				abstractions of interest in a design subject, and a number of new types is to be designed, some of which
				may be considered for reuse. The main concern is the proper choice of abstractions and their expression in
				terms of existing types (some of which may had been specific to the design subject).
			\end{comment}
			
		\subsection{Design elements} \label{s:template-vpv:design-elements}
			\begin{comment}
				Design entities: class, interface, power type, data type, object, attribute, method, association class, template,
				and namespace.
				Design relationships: association, generalization, dependency, realization, implementation, instance of,
				composition, and aggregation.
				Design attributes: name, role name, visibility, cardinality, type, stereotype, redefinition, tagged value,
				parameter, and navigation efficiency.
				Design constraints: value constraints, relationships exclusivity constraints, navigability, generalization sets,
				multiplicity, derivation, changeability, initial value, qualifier, ordering, static, pre-condition, post-condition,
				and generalization set constraints.
			\end{comment}
		
		\subsection{Example languages} \label{s:template-vpv:example-languages}
			\begin{comment}
				UML class diagrams and UML object diagrams (showing objects as instances of their respective classes)
				(OMG [B28]). Lattices of types and references to implemented types are commonly used as supplementary
				information.
			\end{comment}
			
	\section{Dependency viewpoint} \label{s:template-vpv:dependency-viewpoint}
		\begin{comment}
			The Dependency viewpoint specifies the relationships of interconnection and access among entities. These
			relationships include shared information, order of execution, or parameterization of interfaces.
		\end{comment}
		
		\subsection{Design concerns} \label{s:template-vpv:design-concerns}
			\begin{comment}
				A Dependency view provides an overall picture of the design subject in order to assess the impact of
				requirements or design changes. It can help maintainers to isolate entities causing system failures or
				resource bottlenecks. It can aid in producing the system integration plan by identifying the entities that are
				needed by other entities and that must be developed first. This description can also be used by integration
				testing to aid in the production of integration test cases.
			\end{comment}