\chapter{Details of the Level Test Case} \label{chp:}
	\begin{comment}
		Introduce the following subordinate sections. Describe each test case and its unique identifier,
		objectives, outcomes, environmental needs, and special procedures.
	\end{comment}

\section{Test case identifier} \label{s::test-cast-identifier}
	\begin{comment}
		Describe the unique identifier needed by each test case so that it can be distinguished from all other
		test cases. An automated tool may control the generation of the identifiers.
	\end{comment}

\section{Objective}  \label{s::objective}
	\begin{comment}
		Identify and briefly describe the special focus or objective for the test case or a series of test cases.
		This is much more detailed than the Test Items in the Level Test Plan. It may include the risk or
		priority for this particular test case or series of test cases.
	\end{comment}
	
\section{Inputs}  \label{s::inputs}
	\begin{comment}	
		Specify each input required to execute each test case. Some inputs will be specified by value (with
		tolerances where appropriate), whereas others, such as constant tables or transaction files, will be
		specified by name. Identify all appropriate databases, files, terminal messages, memory resident areas,
		and values passed by the operating system.
		Specify all required relationships between inputs (e.g., timing).
	\end{comment}

\section{Outcome(s)}  \label{s::outcome}
	\begin{comment}
		Specify all outputs and the expected behavior (e.g., response time) required of the test items. Provide
		the exact value(s) (with tolerances where appropriate) for each required output and expected behavior.
		This section is not required for self-validating tests.
	\end{comment}

\section{Environmental needs}  \label{s::environmental-needs}
	\begin{comment}
		Describe the test environment needed for test setup, execution, and results recording. This section is
		commonly documented per scenario or group of scenarios. It may be illustrated with one or more
		figures showing all of the components and where they interact. This section is only needed if it
		provides more information than the Level Test Plan or if there have been changes since the Level Test
		Plan was developed.
	\end{comment}

\section{Hardware}  \label{s::hardware}
	\begin{comment}
		Specify the characteristics and configuration(s) of the hardware required to execute this test case.
	\end{comment}
	
\section{Software}  \label{s::software}
	\begin{comment}
		Specify all software configuration(s) required to execute this test case. This may include system
		software such as operating systems, compilers, simulators, and test tools. In addition, the test item may
		interact with application software.
	\end{comment}

\section{Other}  \label{s::other}
	\begin{comment}
		Specify any other requirements not yet included [e.g., unique facility needs, specially trained
		personnel, and third-party provided environment(s)] if there are any.
	\end{comment}
	
\section{Special procedural requirements}  \label{s::special-procedural-requirements}
	\begin{comment}
		Describe any special constraints on the Level Test Procedures that execute this test case such as preand
		post-conditions and/or processing. This section may reference the use of automated test tools. This
		section provides exceptions and/or additions to the Level Test Procedures, not a repeat of any of the
		information contained in the procedures.
	\end{comment}

\section{Intercase dependencies}  \label{s::intercase-dependencies}
	\begin{comment}
		List the identifiers of test cases that must be executed prior to this test case. Summarize the nature of
		the dependencies. If test cases are documented (in a tool or otherwise) in the order in which they need
		to be executed, the Intercase Dependencies for most or all of the cases may not be needed
	\end{comment}
	