\section{\gls{cuda} Feature} \label{s:system-features:cuda-feature}
	\begin{comment}
		$<$Don’t really say “System Feature 1.” State the feature name in just a few 
		words.$>$
	\end{comment}
	
\subsection*{Description and Priority}
	\begin{comment}
		$<$Provide a short description of the feature and indicate whether it is of 
		High, Medium, or Low priority. You could also include specific priority 
		component ratings, such as benefit, penalty, cost, and risk (each rated on a 
		relative scale from a low of 1 to a high of 9).$>$
	\end{comment}
	\textsc{Priority:} High \\
	Library should provide basic computation of matrix--vector product using \glsdesc{CRS} and \glsdesc{ELL} storage formats. Computations should make use of NVIDIA graphics card using \gls{cuda} library routines.
\subsection*{Stimulus/Response Sequences}
	\begin{comment}
		$<$List the sequences of user actions and system responses that stimulate the 
		behavior defined for this feature. These will correspond to the dialog elements 
		associated with use cases.$>$
	\end{comment}
	
	\stimresp
	{User wants to compute matrix--vector dot product using \gls{CRS} or \gls{ELL} on NVIDIA graphics card.}
	{Library computes matrix--vector dot product using \gls{CRS} or \gls{ELL} formats on NVIDIA graphics card.}
\subsection*{Functional Requirements}
	\begin{comment}
		$<$Itemize the detailed functional requirements associated with this feature.  
		These are the software capabilities that must be present in order for the user 
		to carry out the services provided by the feature, or to execute the use case.  
		Include how the product should respond to anticipated error conditions or 
		invalid inputs. Requirements should be concise, complete, unambiguous, 
		verifiable, and necessary. Use “TBD” as a placeholder to indicate when necessary 
		information is not yet available.$>$
		
		$<$Each requirement should be uniquely identified with a sequence number or a 
		meaningful tag of some kind.$>$
		
		REQ-1:	REQ-2:
	\end{comment}
	
	\begin{functional}{Computing matrix--vector dot product using \gls{CRS}}{Medium}{					\ref{fn:common-feature:transform-mm2crs}}
		\label{fn:cuda-feature:compute-crs}
		\term{DESC}{Library should be able to compute matrix--vector dot product using \gls{CRS} sparse matrix format. Calculations should use NVIDIA graphics card using \gls{cuda} Toolkit. As the output of the computation it is assumed to get the result vector and execution time in as a factor of seconds.} 
		\term{RAT}{In order to get functionality to calculate matrix--vector dot product using \glsdesc{CRS} format.}
	\end{functional}
	
	\clearpage
	
	\begin{functional}{Computing matrix--vector dot product using \gls{ELL}}{Medium}{					\ref{fn:common-feature:transform-mm2ell}}
		\label{fn:cuda-feature:compute-ell}
		\term{DESC}{Library should be able to compute matrix--vector dot product using \gls{ELL} sparse matrix format. Calculations should use NVIDIA graphics card using \gls{cuda} Toolkit. As the output of the computation it is assumed to get the result vector and execution time in as a factor of seconds.} 
		\term{RAT}{In order to get functionality to calculate matrix--vector dot product using \glsdesc{ELL} format.}
	\end{functional}
	
